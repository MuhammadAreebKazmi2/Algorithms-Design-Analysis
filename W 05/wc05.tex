\documentclass[a4paper]{exam}
\usepackage[margin=1in]{geometry}
\usepackage{amsmath, amsfonts}
\usepackage{pythonhighlight}
\usepackage{graphicx} %package to manage images
\graphicspath{ {./images/} }
\usepackage{hyperref}
\usepackage{tabularx}
\usepackage{colortbl}
\usepackage{amsmath}

\printanswers
\begin{document}
\begin{center}
{\Large \textbf{Spring 2024}}\vspace{1.0em}\\
{\Large \textbf{CS 412 (Algorithms: Design and Analysis)}}\vspace{1.0em}\\
{\Large \textbf{Weekly Challenge 05 }}\vspace{1.0em}\\
{\Large Announced: Friday, February 16, 2024.}\\
\vspace{.25em}
{\Large Deadline: Friday, February 23, 2024 (11:59 pm PKT).}\\ 
\vspace{.3em}
{\Large Total marks: 1.}
\vspace{.5em}\\
\end{center}
\textbf{Instructions}: Submit \textbf{individually} your solution as a PDF with the file name as your $studentID.pdf$; typeset in LaTeX. You must submit your solution on Canvas.

\centerline{\rule{.7\textwidth}{1pt}}
\begin{questions}
\question[1]
Applying the standard mathematical method to define an algorithm for matrix multiplication gives an $O(n^3)$ solution. Given two $nxn$ matrices, a brute force solution requires $n^3$ arithmetic operations. Just like fast integer multiplication, a solution exists to reduce the time complexity of matrix multiplication. For example, Strassen's algorithm completes matrix multiplication in $O(n^2.81)$. Start by identifying a fast matrix multiplication algorithm with complexity better than $O(n^3)$. Use the algorithm to compute the matrix product: \\
\[ 
     \begin{pmatrix}
      1 & 3\\ 
      6 & 5
    \end{pmatrix}
     \begin{pmatrix}
      a & b\\ 
      c & d
    \end{pmatrix}
\]
Clearly list all the steps of the solution.

Next, compare the dimension of matrix where your algorithm outperforms the brute force algorithm.
% Add your solution here
\begin{solution}
We will go by the hint of Strassen's algorithm \cite{CME323-Lec3} which has the complexity $O(n^{2.81})$

By following the steps below, we will implement the algorithm:
\begin{itemize}
  \item[1] Splitting the matrices $A$ and $B$ where the two 2x2 matrices mentioned above are $A$ and $B$ respectively.
  \[A =     
    \begin{pmatrix}
    1 & 3\\ 
    6 & 5
  \end{pmatrix}
   =      \begin{pmatrix}
    a_{11} & a_{12}\\ 
    a_{21} & a_{22}
  \end{pmatrix}
    \] \\
    \[ B = 
    \begin{pmatrix}
      a & b\\ 
      c & d
    \end{pmatrix} = 
    \begin{pmatrix}
          b_{11} & b_{12}\\ 
          b_{21} & b_{22}
        \end{pmatrix}
    \] \\
  \item[2] Then we compute the matrices by breaking the two nxn matrices into smaller sub-multiplications of $(n/2)x(n/2)$ matrices products: \\
  \[ P_1 = (1+5) \times (a+d) = 6(a+d) \]
  \[ P_2 = (6+5) \times a = 11a \] 
  \[ P_3 = 1 \times (b-d) = b-d \]
  \[ P_4 = 5 \times (c-a) = 5(c-a) \]  
  \[ P_5 = (1+3) \times d = 4d \]
  \[ P_6 = (6-1) \times (a+b) = 5(a+b) \]
  \[ P_7 = (3-5) \times (c+d) = -2(c+d) \]

  \item[3] By combining the sub-products following the algorithm, we will compute the final product matrix C:
  \[c_{11} = P_1 + P_4 - P_5 + P_7 = 6(a+d) + 5(c-a) - 4d - 2(c+d)\]  
  \[c_{12} = P_3 + P_5 = (b - d) + 4d\] 
  \[c_{21} = P_2 + P_4 = 11a + 5(c-a)\] 
  \[c_{22} = P_1 - P_2 + P_3 + P_6 = 6(a+d) - 11a + (b - d) + 5(a+b)\]

  \item[4] Now by combining the resulting submatrices to obtain the final matrix $C$ in the form:
  \[C = \begin{pmatrix}
    c_{11} & c_{12}\\ 
    c_{21} & c_{22}
  \end{pmatrix}\] 

  \[ C = \begin{pmatrix}
    6(a+d) + 5(c-a) - 4d - 2(c+d) & (b - d) + 4d\\ 
    11a + 5(c-a) & 6(a+d) - 11a + (b - d) + 5(a+b)
  \end{pmatrix}\]

  Simplifying the matrix C leads to
  \[ C = \begin{pmatrix}
    a + 3c & b  + 3d\\ 
    6a + 5c & 6b + 5d   
  \end{pmatrix}\]
\end{itemize}

If we compare the Strassen's algorithm with brute force, the Strassen algorithm will have a step ahead of brute force when $n=2$ for the nxn matrices since the time complexity of Strassen is $O(n^{log_{2}7)}$ while the brute force has complexity $O(n^3)$. However, we see more clear difference when $n \geq 30$ and the difference between the two algorithms is vast, and it seems more efficient as the value of $n$ increases.

\end{solution}


\end{questions}

\bibliographystyle{plain}
\bibliography{ref}
\end{document}
