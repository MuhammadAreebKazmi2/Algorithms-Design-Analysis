\documentclass[a4paper]{exam}
\usepackage[margin=1in]{geometry}
\usepackage{amsmath}
\usepackage{geometry}
\usepackage{hyperref}
\usepackage{pythonhighlight}
\usepackage{listings}
\usepackage{xcolor}
\usepackage{graphicx}

\definecolor{codegreen}{rgb}{0,0.6,0}
\definecolor{codegray}{rgb}{0.5,0.5,0.5}
\definecolor{codepurple}{rgb}{0.58,0,0.82}
\definecolor{backcolour}{rgb}{0.95,0.95,0.92}

\lstdefinestyle{mystyle}{
    backgroundcolor=\color{backcolour},   
    commentstyle=\color{codegreen},
    keywordstyle=\color{magenta},
    numberstyle=\tiny\color{codegray},
    stringstyle=\color{codepurple},
    basicstyle=\ttfamily\footnotesize,
    breakatwhitespace=false,         
    breaklines=true,                 
    captionpos=b,                    
    keepspaces=true,                 
    numbers=left,                    
    numbersep=5pt,                  
    showspaces=false,                
    showstringspaces=false,
    showtabs=false,                  
    tabsize=2
}

\lstset{style=mystyle}
\printanswers

\begin{document}
\begin{center}
{\Large \textbf{Spring 2024}}\vspace{1.0em}\\
{\Large \textbf{CS 412 (Algorithms: Design and Analysis)}}\vspace{1.0em}\\
{\Large \textbf{Weekly Challenge 07: Maximum Flows}}\vspace{1.0em}\\
{\Large Announced: Friday, March 1, 2024.}\\
\vspace{.25em}
{\Large Deadline: Friday, March 15, 2024 (11:59 pm PKT).}\\ 
\vspace{.3em}
{\Large Total marks: 2.}
\vspace{.5em}\\
\end{center}
\textbf{Instructions}: Submit \textbf{individually} your solution as a PDF with the file name as your $studentID.pdf$; typeset in LaTeX. You must submit your solution on Canvas.

\centerline{\rule{.7\textwidth}{1pt}}
\begin{questions}
\question[1]
  We are going to write a class \pyth{MaxFlow} in a file \texttt{maxflow.py}. Objects of the class will take in a flow network and compute its maximum flow. Your code will be tested by \pyth{pytest} using the file, \texttt{test\_maxflow.py}, given in \texttt{WC7\_MaxFlows.zip}. To test your implementation of \pyth{MaxFlow}, open the directory containing \texttt{test\_maxflow.py} and \texttt{maxflow.py} in the terminal, and run the following command:
\begin{lstlisting}[language=bash]
pytest test_maxflow.py
\end{lstlisting}
  \textbf{TASKS}:
  \begin{parts}
  \part The \pyth{__init__} method of \pyth{MaxFlow} takes as argument a \pyth{file} object which points to a file containing a flow network, $G$, in the following format.
    \begin{itemize}
    \item The first line contains the number of edges, $e$.
    \item Each of the next $e$ lines contains an edge in the format $u\ v\ c$ where $(u,v)$ is an edge in $G$ with capacity $c$.
    \item The source vertex is always named, $s$, and the sink vertex is always named, $t$.
    \item All capacities are integral and positive.
    \end{itemize}
  \part The \pyth{get_value} method returns the value of the maximum flow on $G$.
  \part The \pyth{get_flow} method returns the maximum flow on $G$ as a \pyth{dict} object. A key in the returned \pyth{dict} object is an edge, $(u,v)$ in the maximum flow and the value is the amount of flow along $(u,v)$. Only edges with non-zero flow are included.
  \part Ensure that all tests pass by running \pyth{pytest} locally.
  \part Do not include any external packages except \pyth{networkx} for the possible use of \href{https://networkx.org/documentation/stable/reference/classes/digraph.html}{\pyth{networkx.DiGraph}} to store and operate on the flow network.
  \part You may modify the error messages in \texttt{test\_maxflow.py} to convey more information if you wish, but you may not alter any other functionality in it.
  \end{parts}

  
  \begin{solution}
    \begin{lstlisting}[language=Python, caption={Python code for MaxFlow class with Ford-Fulkerson algorithm.}, label={lst:maxflow}]
      import networkx as nx
      
      def _dfs(graph, source, sink):
          """
          Depth-First Search (DFS) function to find a path from the source to the sink in the graph.
          """
          stack, visited = [(source, [source])], {source}
      
          while stack:
              u, path = stack.pop()
              if u == sink:
                  return path
              for v in graph[u]:
                  if v not in visited and graph[u][v]['capacity'] > 0:
                      visited.add(v)
                      stack.append((v, path + [v]))
          return []
      
      class MaxFlow:
          def __init__(self, file) -> None:
              """
              Initialize the MaxFlow class with a flow network provided in the given file.
              """
              self.graph = nx.DiGraph()
              self._read_flow_network(file)
              self.residual_graph = self.graph.copy()  # Create a copy of the graph for the residual network
      
          def _read_flow_network(self, file):
              """
              Read the flow network from the file and populate the graph.
              """
              for line in file:
                  u, v, capacity = line.split()
                  self.graph.add_edge(u, v, capacity=int(capacity))
      
          def get_value(self):
              """
              Compute the maximum flow in the flow network using the Ford-Fulkerson algorithm.
              """
              source, sink, max_flow = 's', 't', 0
      
              path = _dfs(self.graph, source, sink)
      
              while path:
                  bottleneck_capacity = float('inf')
      
                  # Find the bottleneck capacity along the augmenting path
                  for index in range(len(path) - 1):
                      u, v = path[index], path[index + 1]
                      capacity = self.residual_graph[u][v]['capacity']
                      bottleneck_capacity = min(bottleneck_capacity, capacity)
      
                  # Update the residual capacities and augment the flow along the path
                  for index in range(len(path) - 1):
                      u, v = path[index], path[index + 1]
                      residual_capacity = self.residual_graph[u][v]['capacity'] - bottleneck_capacity
      
                      # Update the residual capacities in the residual graph
                      if not residual_capacity:
                          self.residual_graph.remove_edge(u, v)
                      else:
                          self.residual_graph[u][v]['capacity'] = residual_capacity
      
                      # Add reverse edges and update their capacities
                      if not self.residual_graph.has_edge(v, u):
                          self.residual_graph.add_edge(v, u, capacity=0)
                      self.residual_graph[v][u]['capacity'] += bottleneck_capacity
      
                  # Update the maximum flow
                  max_flow += bottleneck_capacity
      
                  # Find the next augmenting path
                  path = _dfs(self.residual_graph, source, sink)
      
              return max_flow
      
          def get_flow(self):
              """
              Compute the flow on each edge in the flow network.
              """
              flow_dict = {}
              for u, v in self.graph.edges():
                  if self.residual_graph.has_edge(u, v):
                      # Calculate flow by subtracting residual capacity from original capacity
                      flow = self.graph[u][v]['capacity'] - self.residual_graph[u][v]['capacity']
                  else:
                      flow = self.graph[u][v]['capacity']
      
                  # Add non-zero flows to the dictionary
                  if flow > 0:
                      flow_dict[(u, v)] = flow
              return flow_dict
      \end{lstlisting}
      
  \end{solution}
\end{questions}
\end{document}

%%% Local Variables:
%%% mode: latex
%%% TeX-master: t
%%% End:
